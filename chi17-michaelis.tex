\documentclass{sigchi}


\CopyrightYear{2017}
\setcopyright{acmcopyright}
 %\setcopyright{rightsretained}
%\setcopyright{usgov}
%\setcopyright{usgovmixed}
%\setcopyright{cagov}
%\setcopyright{cagovmixed}
% DOI
\doi{http://dx.doi.org/10.475/123_4}
% ISBN
\isbn{123-4567-24-567/08/06}
%Conference
\conferenceinfo{CHI'17,}{May 06--11, 2017, Denver, CO, USA}
%Price
\acmPrice{\$15.00}

% Use this command to override the default ACM copyright statement
% (e.g., for preprints).  Consult the conference website for the
% camera-ready copyright statement.

%% HOW TO OVERRIDE THE DEFAULT COPYRIGHT STRIP --
%% Please note you need to make sure the copy for your specific
%% license is used here!
% \toappear{
% Permission to make digital or hard copies of all or part of this work
% for personal or classroom use is granted without fee provided that
% copies are not made or distributed for profit or commercial advantage
% and that copies bear this notice and the full citation on the first
% page. Copyrights for components of this work owned by others than ACM
% must be honored. Abstracting with credit is permitted. To copy
% otherwise, or republish, to post on servers or to redistribute to
% lists, requires prior specific permission and/or a fee. Request
% permissions from \href{mailto:Permissions@acm.org}{Permissions@acm.org}. \\
% \emph{CHI '16},  May 07--12, 2016, San Jose, CA, USA \\
% ACM xxx-x-xxxx-xxxx-x/xx/xx\ldots \$15.00 \\
% DOI: \url{http://dx.doi.org/xx.xxxx/xxxxxxx.xxxxxxx}
% }

% Arabic page numbers for submission.  Remove this line to eliminate
% page numbers for the camera ready copy
% \pagenumbering{arabic}

% Load basic packages
\usepackage{balance}       % to better equalize the last page
\usepackage{graphics}      % for EPS, load graphicx instead 
\usepackage[T1]{fontenc}   % for umlauts and other diaeresis
\usepackage{txfonts}
\usepackage{mathptmx}
\usepackage[pdflang={en-US},pdftex]{hyperref}
\usepackage{color}
\usepackage{booktabs}
\usepackage{textcomp}
\usepackage{soul}

% Some optional stuff you might like/need.
\usepackage{microtype}        % Improved Tracking and Kerning
% \usepackage[all]{hypcap}    % Fixes bug in hyperref caption linking
\usepackage{ccicons}          % Cite your images correctly!
% \usepackage[utf8]{inputenc} % for a UTF8 editor only

% If you want to use todo notes, marginpars etc. during creation of
% your draft document, you have to enable the "chi_draft" option for
% the document class. To do this, change the very first line to:
% "\documentclass[chi_draft]{sigchi}". You can then place todo notes
% by using the "\todo{...}"  command. Make sure to disable the draft
% option again before submitting your final document.
\usepackage{todonotes}

% Paper metadata (use plain text, for PDF inclusion and later
% re-using, if desired).  Use \emtpyauthor when submitting for review
% so you remain anonymous.
\def\plaintitle{Someone to Read with: Design of and Experiences with \\ an In-Home Learning Companion Robot for Reading}
\def\plainauthor{}
\def\emptyauthor{}
\def\plainkeywords{Human-robot interaction; interest development; reading education; educational robots; design probes}
\def\plaingeneralterms{Documentation, Standardization}

% llt: Define a global style for URLs, rather that the default one
\makeatletter
\def\url@leostyle{%
  \@ifundefined{selectfont}{
    \def\UrlFont{\sf}
  }{
    \def\UrlFont{\small\bf\ttfamily}
  }}
\makeatother
\urlstyle{leo}

% To make various LaTeX processors do the right thing with page size.
\def\pprw{8.5in}
\def\pprh{11in}
\special{papersize=\pprw,\pprh}
\setlength{\paperwidth}{\pprw}
\setlength{\paperheight}{\pprh}
\setlength{\pdfpagewidth}{\pprw}
\setlength{\pdfpageheight}{\pprh}

% Make sure hyperref comes last of your loaded packages, to give it a
% fighting chance of not being over-written, since its job is to
% redefine many LaTeX commands.
\definecolor{linkColor}{RGB}{6,125,233}
\hypersetup{%
  pdftitle={\plaintitle},
% Use \plainauthor for final version.
  pdfauthor={\plainauthor},
%  pdfauthor={\emptyauthor},
  pdfkeywords={\plainkeywords},
  pdfdisplaydoctitle=true, % For Accessibility
  bookmarksnumbered,
  pdfstartview={FitH},
  colorlinks,
  citecolor=black,
  filecolor=black,
  linkcolor=black,
  urlcolor=linkColor,
  breaklinks=true,
  hypertexnames=false
}

% create a shortcut to typeset table headings
% \newcommand\tabhead[1]{\small\textbf{#1}}

% End of preamble. Here it comes the document.
\begin{document}

\title{\plaintitle\vspace{56pt}}

%\numberofauthors{2}
%\author{%
%  \alignauthor{Leave Authors Anonymous\\
%    \affaddr{for Submission}\\
%    \affaddr{City, Country}\\
%    \email{e-mail address}}\\
%  \alignauthor{Leave Authors Anonymous\\
%    \affaddr{for Submission}\\
%    \affaddr{City, Country}\\
%    \email{e-mail address}}\\
%  %\alignauthor{Leave Authors Anonymous\\
%   % \affaddr{for Submission}\\
%    %\affaddr{City, Country}\\
%    %\email{e-mail address}}\\
%}

\maketitle

\begin{abstract}
  The development of literacy and reading proficiency is a building block of lifelong learning that must be supported both in the classroom and at home. While the promise of interactive learning technologies has widely been demonstrated, little is known about how an interactive robot might play a role in this development. We used eight design features based on recommendations from interest-development and human-robot-interaction literatures to design an in-home learning companion robot for children aged 11--12. The robot was used as a technology probe to explore families' ($N=8$) habits and views about reading, how a reading technology might be used, and how children perceived reading with the robot. Our results indicate reading with the learning companion to be a way to socially engage with reading, which may promote the development of reading interest and ability. We discuss design and research implications based on our findings.
  %145 words, 150 max
\end{abstract}

\category{H.1.2.}{\textbf{Models and Principles}}{User/Machine Systems}{\textit{---human factors, software psychology}}
\category{H.5.2.}{\textbf{Information Interfaces and Presentation}}{User Interfaces}{\textit{---evaluation/methodology, user-centered design}}

\keywords{\plainkeywords}

\section{Introduction}

\begin{figure}[t]
	\centering
	\includegraphics[width=1\columnwidth]{figures/childWithRobotTeaser_2}
	\caption{A study participant reads James and the Giant Peach to Minnie, our learning companion robot.}~\label{fig:teaser}\vspace{-16pt}
\end{figure}

Reading has long had an important and established place in education, and developing literacy and reading proficiency is considered a basic and essential element of learning \cite{McCormick:1994,Freire:1983}. While classroom instruction for younger readers is an important part of this process, practicing and engaging with reading at home is also crucial for these students \cite{Baker:1997}. With drastic technological advances in the last decade, using robots as educational agents to support learning both in classrooms and in the home has become increasingly viable \cite{Benitti:2012}. These educational robots can act as learning companions, tutors, or instructional agents, providing individually tailored instruction to children \cite{Miller:2008}. When used in the home, robots promise to support student learning and skill-building to supplement classroom instruction.

One type of instructional support that has proven beneficial to both learning outcomes and student engagement is the development of \textit{interest} in a domain \cite{Hidi:2006}, and supporting this development can be particularly useful for young readers \cite{Jones:2011}. However, little is known about how an in-home learning companion robot might provide students with support for interest and engagement in reading. In this paper, we present work aimed to explore how young readers currently engage in reading at home and how they imagine technologies such as robots could enhance this activity; to observe how they interact with a robot designed as an in-home learning companion for reading; and to explore their beliefs about how they might use and benefit from a robot as a learning companion. To pursue these goals, we developed an in-home learning companion robot, named ``Minnie,'' to use as a technology probe \cite{Hutchinson:2003}.
% The name Minnie is after the Owl of the Roman Goddess Minerva.

In the following sections, we report on the use of Minnie, a robot designed with guidelines offered by prior research in interest development and human-robot interaction, as a probe to gain a better understanding of how a learning companion robot may be used in a child's home, and how our design features will be perceived by the child and their family. After providing a brief background on the research that guided our work, we describe how the robot was developed and used as a technology probe. We then detail our data collection methods and results, discuss the implications of these results, and finally conclude with theoretical and practical contributions to both interest development and human-robot interaction.

\section{Background}
Our work is inspired by prior research that has highlighted the educational promise of robots in the home and their role as social agents compared to other learning technologies. Han et al. \cite{Han:2005} were among the first to report on testing an educational robot for children in their homes. They found that a home robot promoted engagement, interest, and learning in English and that users considered the robot to be friendlier than other learning media. These benefits may be due to the robot's ability to better connect socially with its user and perform more appropriate behavioral strategies than other computerized agents \cite{Brown:2013}, as social interactions between humans have been found to positively influence interest and learning in a task \cite{Sansone:2005}. Other studies in human-robot interaction have found robots to be more engaging and life-like \cite{Kiesler:2008}; appealing, perceptive and helpful \cite{Wainer:2007}; and positive and natural \cite{Bainbridge:2011} than other computerized agents. This social nature of human-robot interaction motivates our exploration of robots as learning companions rather than other computerized agents or alternative learning technologies for reading.

Our investigation draws on research in interest development as well as in human-robot interaction to identify guidelines for supporting student engagement and interest in a reading task. Interest-development researchers distinguish between \textit{situational interest}, a psychological state characterized by increased focus and attention, and \textit{individual interest}, a predisposition to reengage with specific content \cite{Krapp:1999}. Hidi and Renninger's  \cite{Hidi:2006} Four-phase Model of Interest Development further refines the situational-individual interest distinction into four progressive phases of interest: Triggered Situational, Maintained Situational, Emerging Individual, and Well-Developed Individual. Each phase in the Four-phase model is considered to be sequential and distinct, with each subsequent phase building off the previous. Interest is characterized and developed by increasing positive affect, value, and knowledge for the domain. \cite{Renninger:2011}   

Promoting situational interest by successfully \textit{triggering} (or catching) and then \textit{maintaining} (or holding) a student's interest during successive interactions can promote the development and maintainance of individual interest \cite{Hidi:2006, Mitchell:1993}. Guidelines for catching and holding interest from this area of research include providing students with autonomy in choosing educational materials \cite{Jones:2011}; setting and monitoring reading goals \cite{Cabral:2015}; providing materials that align with student topic interests \cite{Ainley:2002} and that students believe they have the ability to be successful with; providing social partners who demonstrate interest in the activity \cite{Sansone:2005}; and reading out-loud to a social partner \cite{Rasinski:2003}. 

Findings from human-robot interaction studies also provide several guidelines for increasing user engagement in human-robot interaction, including the robot establishing and maintaining direct eye contact when speaking to its user \cite{Mutlu:2011}; providing tailored recommendations for content \cite{Lim:2013}; expressing empathy \cite{Leite:2012}; and making reference to previous interactions \cite{Leite:2009}. By utilizing these design guidelines, we believe a robotic agent may be a powerful tool for supporting interest and engagement in reading for young readers. Several specific features can be incorporated into a learning companion robot in order to promote student interest and engagement in reading. The next section details how the design process for our technology probe integrates these features.

% TODO: Making this text into a Background section is one of the biggest changes I made, which the readers will expect to see. As a background section, it could be extended a little bit to make sure that you are covering your bases on other related work from HRI, CHI, etc. so that if one of your reviewers did or know work that's similar, they don't go ballistic on us for having missed it.

% NOTE: I thought about adding a few citations about promoting social interaction though the physical design.  Duffy, 2003: Anthropomorphism and the social robot http://dx.doi.org/10.1016/S0921-8890(02)00374-3 ; and Tung, 2016: Child Perception of Humanoid Robot Appearance and Behavior doi: 10.1080/10447318.2016.1172808.  Martini 2015: Minimal physical features required for social robots. My hesitation is that we are sort of locked into only having 8 design features, since that is in our abstract.  Thoughts?

%Added some Interest development theory to support classifying the children using the four-phase model terminology.

\section{DESIGN AND USE OF THE TECHNOLOGY PROBE}
 For this study, we first developed a learning companion robot, Minnie, based on the design guidelines discussed in the previous section. Following design research practices from prior work, such as work by Odom et. al \cite{Odom:2012}, we then conducted short in-home visits with families to use the learning companion robot as a \textit{technology probe} \cite{Hutchinson:2003} for discussion and reflection on how a reading technology might impact current reading habits in each home. Hutchinson et al. \cite{Hutchinson:2003} describe the technology probe approach as a method with three goals---understanding the user's needs and feelings about the technology, field-testing a technology, and reflection on new possibilities for the technology---that reflect our aims for this study. At this stage of development, our prior understanding of the intended user may not be sufficient, which may limit our view of how and what the learning companion robot should do. The introduction of the learning companion into the home may uncover unknown user needs and expectations as well as inspire new thoughts on and possibilities for the learning companion's use. We intend to explore these ideas to inform design modifications to the learning companion robot prior to empirical testing. Thus, our field testing of the robot is conducted concurrent to deep exploration of the design space through the eyes of the user. In essence, we must study the user, the technology, and their interaction all at the same time.

% TODO: This last sentence is great, and you explain what you mean by it in the second half of the next paragraph, but perhaps make it a bit more clear?
 
 We began the in-home visit by asking one child (referred to after as the \textit{main child}) to complete a short survey on individual interest in reading, followed by an interview (pre-) with the family. After the pre-interview, we introduced our learning companion robot to the family and asked the main child in the home to interact with it for a short period of time (approximately 30 minutes). Following the interaction, we asked the main child to complete another short survey, this time about their situational interest during the interaction. Finally, we completed a second interview (post-) with the family. In the spirit of using the technology as a probe, we did not limit the discussion or interaction with the robot to our preconceived ideas about how, when and where the learning companion should and would be used. Rather, we allowed families to explore ideas about how they would like to or would use this or another technology for reading. The following paragraphs describe this data collection method and subsequent analytic approach in more detail.
 
\subsection{Development of the Robot Platform}
Our initial design of the learning companion robot was guided by a desire to leverage the power of social interaction to promote interest and engagement in reading. We determined that a small, desktop humanoid robot with multiple methods of interaction and non-verbal expression would be appropriate for interaction with a child over the activity of reading. We also felt that the robot should be built on a simple, modifiable, and inexpensive platform in order to model the type of constraints inherent in an affordable commercially available robot. This intuition led us to create the robot using a modified version of a freely available 3D-printable robot design from Hello Robo\footnote{Hello Robo: \href{http://www.hello-robo.com/}{http://www.hello-robo.com/}} called Maki. A more expensive pre-made robot such as the popular Nao platform by SoftBank Robotics,\footnote{SoftBank Robotics: \href{https://www.ald.softbankrobotics.com/en}{https://www.ald.softbankrobotics.com/en}} while possibly offering a more reliable and polished experience, is not flexible in its physical and hardware design, and the cost would preclude many families from actual use in the their homes. The original robot from Hello Robo stands 13.5 inches and has a static torso without any movable appendages (see Figure \ref{fig:complete-system}). The head and neck of the robot contain five servo controlled motions, including two closing eyelids, a pair of eyes that move laterally together, and vertical and horizontal head movements. These motion options allowed us to create specific robot behaviors to provide a life-like feel (e.g., blinking eyes) and non-verbal communication (e.g., looking directly at the child). 

To perform the necessary computation and facilitate human-robot communication, we modified the original robot design by Hello Robo by adding four pieces of hardware to the robot. First, we integrated a Raspberry Pi 3 Model B \footnote{Raspberry Pi: \href{https://www.raspberrypi.org/}{https://www.raspberrypi.org/}} microcomputer running the \hl{xx} operating system to perform the sensing, actuation, and computation necessary for the robot's functioning and interaction with the child. Integration of the microcomputer also required us to modify the 3D model of the robot, and thus we extended the back wall of the robot's torso in the shape of a small ``backpack'' that would house the computer and provide us or the end users with easy access to the hardware. Second, we embedded a speaker in each of the ear pieces to allow the robot to utilize text-to-speech responses to the child. Third, we included an RFID reader in the lower front of the robot to allow the child to respond to the robot with four color coded, pre-programmed RFID cards. Roughly, the cards allowed the child to \textit{say} ``yes/continue,'' ``no/stop,'' ``pause,'' and ``repeat'' to the robot. Fourth, we added a USB camera to achieve two purposes: allowing video input to process facial recognition, which we used for facial tracking and reading visual tags (AprilTags\footnote{AprilTags: \href{https://april.eecs.umich.edu/wiki/AprilTags}{https://april.eecs.umich.edu/wiki/AprilTags} }) placed within books. Each time an RFID card or an AprilTag is scanned, Minnie blinks and makes a distinct beep to let the child know that the scan was registered. The robot platform allows us to implement the eight design elements we identified from our literature review.

\begin{figure}
	\centering
	\includegraphics[width=\columnwidth]{figures/Complete-Minnie-System.jpg}
	\caption{Image of the learning companion robot, Minnie, used in the study as well as the introductory ``Talking with Minnie'' book, RFID topic and interaction cards, and books augmented with AR tags. The robot's camera is the red circle on the upper torso. The RFID reader is below the camera, and indicated by a red light when the robot is on. }~\label{fig:complete-system}\vspace{-16pt}
\end{figure}

\subsection{Interaction Design of the Robot}

The interaction design of our technology probe drew on design guidelines provided by research in interest development and human-robot interaction to support student engagement and interest in reading, utilizing the eight design elementguidelines 

\begin{enumerate}
\setlength\itemsep{-0.5pt}
\item Providing students with autonomy in choosing educational materials;
\item Setting and monitoring reading goals;
\item Providing materials that align with student topic interests and with which students believe they have the ability to be successful;
\item Providing a social partner that indicate interest in the reading activity;
\item Reading out-loud to a social partner;
\item Having the robot make direct eye contact when speaking;
\item Providing tailored recommendations for content;
\item Making references to previous interactions. 
\end{enumerate}


 In the description of the interaction design of the robot below, we will identify where design features are addressed in parentheses. We begin with a discussion of the general design of the robot interaction, followed by specific interaction details particular to this study.
% TODO: One of the questions the readers are going to have is where these guidelines come from specifically. After reading the rest of the design section, it is probably best to integrate that information here where you list the design features.

\subsubsection{Design of the Interaction Flow}

During the interaction, Minnie, the learning companion robot, uses face recognition to track the child's position and turn its head and eyes toward the child to establish eye contact (design elementf6). However, to prevent long durations of eyeeye ctact from eiciting discomfort, the robot makes idle head and eye movements and engages in gaze aversionnt, and Minnie is programmed trobotMinnie also continually uses teo deliver pre-scripted dialog based on an interaction flow, described below, which we developh iterative design and testing.

\textit{Introduction --} The interaction flow begins with Minnie saying ``hello'' to the child and providing a brief backstory on why the robot is there (i.e., the robot tells the child that she is a reading coach in training). The child is then instructed to scan and read a book, ``Talking with Minnie,'' that was written by the authors of this work, and is used to teach the child how to interact with the robot. This book is written as a story, but the content guides the child through the use of each interactive tool (e.g., how to scan the RFID cards and AprilTags as well as, and they mean), and how to work with Minnie (e.g., read out loud and scan AprilTags). The robot then asks the child to complete a topic- sort activit using 10 topic interest cards (e.g., mystery, art, science, andor sci-fi/fantasy)ith AprilTags. For this activitytopic sort, Minnie the child to scan the topics they like, followed by any topics they do not like. The robotsoftware stores this ion in the user profile for the child and later uses this as part of a book suggestion algorithm. Figure \ref{fig:introduction} show2 illustrates a child working wuring the introduction process.
\begin{figure}
	\centering
	\includegraphics[width=1\columnwidth]{figures/Child-with-Minnie}
	\caption{Image of a child working with Minnie during the introductory process. The child is holding a topic interest card, and has placed the Talking with Minnie book on the floor in front of Minnie. The books available for the child to read appear on the right of the image. An example of a green RFID card and a topic interest card with an April tag are superimposed on the top right of the figure.}
	\label{fig:introduction}
\end{figure}

\textit{Goal Setting --} After completing the initial introduction phase, Minnie begins the process of setting reading goals and selecting bookb. During this process, Minnie informs the child of a goal for reading time that day (design elementfea -- for this study, standardized at 20 minutes in length -- before each reading session, and begins tracking the time spent readng after a book is chosen. The learning companion robot will then check the user profile for a \teurrent book}, a previously read book that has not been finished. If a current book exists, Minnie will remind the child ofthat this was what they were last reading and telthe page number they were on (design elementfeature 8. The child then has the option to continue or not. If theo current book selection, Minnie suggests three books (out of a set of 16f the 16 available) to the child that the childy may be interested in                                    's suggestions for books are based on an algorithm that uses the child's reading ability, interest in reading, usual time spent reading, and topic interests to recommendeddetermine ideal page length, reading level, and topical content of a book for the child (design elementfeature 3). The program then assigns each book listed in the library a score based on the how well they match the recommendationideal, and creates a priority cue. The top three books in the priority cue are offered to the child as suggested titles (design elementfeature 7). The child may chose any book they wish (design elementfeature 1), and scan thean AprilTag placed on the back of the book they choose to indicate their selection. Then the robotsoftware updates the user profile to store the selected bookis as the child's current book;, Minnie confirms the title of the book with the child;, and the child is asked to begin reading on page one, which. This marks the beginning of the reading phase.
% TODO: The reader will want to know how this "algorithm" works to come up with the recommendation. Is it some kind of cost-based search?

\textit{Reading --} In the reading phase, the child is asked to read \textit{out loud} to Minnie (design element 5). Minnie is programmed to make semi-randomized movements during the reading phase, which include blinking, moving its eyes, and small turns of its head in order to appear engaged in the reading (design elementfeature the child reads, they will find AprilTags added to roughly every three to six pages of each book, and the child is instructed to scan these tags when they encounter them. The tag ID is linked to a specific comment, written by our research team, related to what is currently happening in the book (design elementfeature 4). W scan is received by Minnie, it blinks, makes a distinct \textit{beep}, and delivers the comment via text-to-speech. Each time a book tag is scanned, the user profile is updated to store that page as the current page. During the reading phase, the child may continue to read, scan the \textit{yellow} RFID to pause reading, or scan the \textit{red} RFID to quit reading. This phase continues until the child chooses to quit, or scans the last age of the book. At that point, Minnie enters the ``end- reading'' phase.
\textit{End Reading --} The end of the reading phase is triggered by either the child choosing to quit reading or when a book has been completed. In both cases, Minnie will compare the time spent reading to their readg goal previously set (design elementfeature the childy have mettheir reading goal f the day, Minnie will congratulate the childm, and begin the sut- down process. Shuttng down consists of thanking the child for the time spent reading, encouraging them to read more the next day, and telling the child that the learning companion robot is going to sleep. If the reading goal has not been mety have not met their  childm know how much time ithey have remaining to reach their goal, and ask whethef they should continue rding. If the cildy choses to continue rding, Minnie asks if the childy would like t read the same book or a new one, and then reenters thereading phase. Otherwise, Minnie enters the shu down procedure. 

\textit{Updating User Profile --} When a child scans the last page of a book, Minnie will ask the child whether they liked the book. The child can respond with \textit{yes} (green card), \textit{no} (red card), or \textit{a little} (yellow card). This response is stored in the user profile and is used to trigger a recalculation of book suggestion scores. For books the child liked, the topics associated with that book will be given a higher score rating, which will move books with similar topics up the priority cue. The opposite is true for books that the child dislikes. In this way, Minnie takes the child's feedback about the books they've read together into consideration for future book recommendations (design features 7 and 8).

The learning companion robot is designed to be used repeatedly in a child's home, and the interaction can be initiated any time by turning the robot on. By addressing the eight design features identified from previous work, we believe that the interaction with the robot may be capable of promoting situational interest for the reading task. We next discuss specific methods of interaction that were utilized in this study.

\subsection{Technology Probe Setup and Modifications}
A few methodological choices were made in order to best use our learning companion robot as a technology probe during a brief in-home visit. These choices involved strategically placingchoosing the plag the interaction, and interruptin the child during their reading in order to facilitate further exploration of Minnie's features. We also asked the child to behave as if the experimenters were not in the roomthere during the interaction, in order to itate as authentic an ineraction as possible. Only in cases where technical difficulties posed a large barrier to their ability to proceed did the experimenter intervene.

The placement of the robot was chosen after the family was introduced to the learning companion robot. The child was asked where in their home they would likely read with the robot. Of the suggestions made by the child, we asked the parent(s) which of these areas they would be most comfortable with us conducting the robot interaction and chose to work there. In this way, we were able to allow for as authentic of an interaction between the child and robot as possible, while maintaining the comfort and trust of the parent. 

The general design of the learning companion robot interaction was programmed as described above. However, for this technology probe, it was unlikely that the child would encounter many of the features due to the short duration of the interaction. To address this issue, each child was interrupted after approximately 10 minutes of reading and told that they should pretend they were tired of reading the current book. The child scanned the red card to quit reading that book, and waere instred to follow Minnie's prompts. Since the reading goal would not have been met, Minnie would tell the child thatm they had't met their goal, and offer t continue reading. Each child was instructed to chose another book, and continue reading. Each child was allowed to read for a maximum of 30 minutes, before being asked to end their reading session.


\subsection{Participants}
 We recruited families (N=8) from a mid-sized Midwestern city to participate in the study. Most often one parent and one child participated, while one family included two parents, and two families had one younger sibling present. For each family, one main child (mean age = 11.6; age range: 11 - 12; male = 5) was asked to interact directly with the learning companion robot. All other family members were included in pre- and post-interviews.
 

\subsection{Data Sources}
  During the course of our study, we collected data from three major sources. The first type of data source were three separate surveys. One survey was completed by parents as pre-visit questionnaire on their child's reading ability and habits. The other two were completed by the main child: an individual interest in reading survey prior to working with the learning companion robot, and a situational interest survey after. Reliability for the individual and situational interest scales was measured using Cronbach's alpha, with an alpha value of 0.7 or greater indicating, by convention, a reasonably reliable construct \cite{Crocker:2009}. The second data source was from transcriptions of video recorded semi-structured interviews conducted before (pre-) and after (post-) the robot interaction. Both interviews were approximately 20 minutes in length, with the pre-interview (\textit{M} = 18.7 minutes, \textit{SD} = 2.3 minutes) being slightly longer than the post-interview (\textit{M} = \#\#\# minutes, \textit{SD} = \#\#\# minutes). Finally, the third data source comes from coding segments of video recordings made during the robot interaction of about 30 minutes (\textit{M} = \#\#\# minutes, \textit{SD} = \#\#\# minutes)). The following discusses each data source in more detail.
  
\subsubsection{Survey}

  Prior to the in-home visit one parent for each family was asked to complete a questionnaire to collect some information about the main child. We asked the parent to describe, to the best of their knowledge, their child's reading ability, the amount of time their child spends reading non-school work each week, their (the parent's) satisfaction with the amount of time their child spends reading, as well as some basic demographic information (e.g., age and grade).
 
  During the in-home visit, the main child was first asked to complete a 10-item pre-survey to assess the child's individual interest in reading. This survey was developed by rewording items from the Four-Phase Interest in Engineering Survey (FIDES) by Michaelis \& Nathan \cite{Michaelis:2015} with the reading motivation language in the Motivation for Reading Questionnaire by Wigfield and Guthrie \cite{Wigfield:1997}. The survey is comprised of Likert-style items, which asked students to rate their level of agreement with statements using a scale from 1 (\textit{strongly disagree}) to 7 (\textit{strongly agree}). Scores were calculated by averaging a sum of scores from all items to produce scores on the 1 to 7 Likert scale equivalent (\textit{M} = 5.8, \textit{SD} = 0.82). The high Cronbach's alpha ($\alpha$ = 0.81) indicates it has a high internal reliability.
 
  After engaging with the learning companion robot, the child was asked to complete a post-survey with six Likert-style items, using the same 1 to 7 scale, to assess the child's level of catch (2 items) and hold (4 items) situational interest during the reading activity. The survey was based on prior work by Knogler et al. \cite{Knogler:2015}. Scores were averaged for catch (Mean = 5.69, \textit{SD} = 1.25) and hold (\textit{M} = 5.66, \textit{SD} = 0.97) items to produce scores on the 1 to 7 Likert scale equivalent. Cronbach\text{'}s alphas were calculated for the situational interest scale as a whole ($\alpha$ = 0.81), and separately for catch ($\alpha$ = 0.50) and hold ($\alpha$ = 0.64) scales. While the overall reliability of the situational scale was high, the sub-scales had lower than accepted values. Thus, we will use the combined situational interest scale in our analysis.
  
\subsubsection{Interview and Robot Interaction}
  Two sources of video data were analyzed for the study: two interviews with the family (pre- and post-), and the main child's interaction with the robot. For the semi-structured interviews, a pre-determined set of questions was used as a focus for the interaction. However, the researcher pursued ideas from the family members in a conversational style, and encouraged them to elaborate and explain their thinking. The pre- and post- interviews were both transcribed verbatim and segmented by idea\cite{Chi:1997}.
 
  We video recorded a semi-structured pre-interview with the family about their current reading habits (e.g., How often do you read?); what motivates their child to read (e.g., What systems do you have in place to encourage your child to read?); and the family's familiarity, experience, and thoughts about working with technology for reading (e.g., If you had a technology that worked with you while you read, what would it do?). After the pre-interview, the main child was introduced to the robot, and instructed on how to begin working with the robot. Once they were ready to begin, we video recorded the entirety of their interaction. After the robot interaction was completed, and a short situational interest survey was administered, we conducted a second semi-structured interview (post-) with the family. This interview was about the family members' view of the robot as a partner (e.g., Did it feel like Minnie was alive?), their likes and dislikes about the robot interaction (e.g., Are there any things you would change about Minnie?), and whether they thought the robot would be a useful tool for the home (e.g., If the robot were released to the market today, would you be interested in buying one for your child?)
  
  All video data was coded using a Grounded Theory \cite{Glaser:1967,Charmaz:2012} approach. In using a Grounded Theory analysis we begin with an open coding period where informal codes were used to summarize significant idea units.  Emerging themes from this open coding were then identified, defined as codes, and applied to the transcriptions as initial codes. After initial codes were identified and applied, the reliability of these codes was examined by comparing inter-rater reliability on 10\% of the data. The inter-rater reliability was very high between coders for both pre- ($\kappa$ = \#\#\#) and post-interviews ($\kappa$ = 0.67), as well for the robot interaction videos ($\kappa$ = \#\#\#). We then developed axial codes for each data set to further refine our coding scheme into related categories that were organized into \#\#\# major themes. The findings below present details of each of the major themes, and are followed by a discussion of our interpretation of these findings.
 
\section{Findings}
Here we report on the findings from our technology probe during in-homes visits for our eight participating families. We discuss four main themes that emerged from our analysis. We first describe what we found to be the relationship with reading for each of our main children: their habits, ability, and interest phase. We will then detail the three remaining themes that emerged from interaction with the robot. Each theme is supported with evidence from interview and/or robot interaction data. Figure~\ref{fig:figure4} describes each of the major themes of our findings. 

As a technical note: throughout the description of our findings families are numbered 1 through 8 with each member identified as ``MC'' for the main child in the study, ``P'' for the parent, and ``OC'' for another child present during interviews. For example: the main child in family number seven is labeled as ``MC7'', and the parent for family 4 is labeled as ``P4''. The researcher conducting the interviews is labeled as ``R'' throughout. Non-verbal actions, such as nodding their head in agreement, appear between double parenthesis e.g., ((nods head yes)), and paraphrasing is indicated by words surrounded by brackets.

\begin{figure}
	\vspace{2mm}
	\textbf{Summary of Results}\\
	%\vspace{2mm} #### Can't get this to work!
	%\midrule
	\textbf{Relationship with Reading} \\
	Each of the children described their relationship to reading that was consistent with one of the phases of the Four-phase Model of Interest Development. \\
	
	\textbf{Development of Interest}
	Children describe their interaction with the robot as enjoyable, valuable, and capable of improving their ability -- all characteristics of interest development.\\
	
	\textbf{Attribution of Human Characteristics} \\
	Children often attributed human characteristics, including emotions and thought, to the robot.\\
	
	\textbf{Social Companionship} \\
	Children in the study refer to the robot in terms of having \textit{some else there} to read with.\\
	%\vspace{2mm} #### Can't get this to work!
	%\midrule
	\caption{Summary of each of the themes emerging from our analysis.}
	\label{fig:figure4}
\end{figure}

\subsection{Relationship with Reading}
In recruiting for our study we initially intended to enroll children with a variety of interest levels in reading. During a short, online, pre-screening questionnaire, we included a question that asked a parent to estimate the amount of time their child spent on reading each week. We believed this would prove useful for inferring a relative interest level, but realized soon after we began our visits that this estimate was often not very accurate. Thus, we did not make any enrollment selections based on pre-screening. We used pre-interview descriptions of the child's view and habits towards reading to identify what phase of interest in reading they were likely in. While we do not believe that any of our participants are categorized in the first and lowest phase of interest development, \textit{triggered situational interest}, we classified 2 participants (MC1 and MC2) in phase two, \textit{maintained situational interest}, 3 (MC3, MC4, and MC5) in phase three, \textit{emerging individual interest}, and 3 (MC7, MC6, and MC8) in the highest and fourth phase, \textit{well-developed individual interest}. These evaluations were supported by results from each child's FID-RS individual interest survey score. We found that our phase two readers had the lowest and second lowest scores (4.5 and 5.4), the emerging individual interest readers fell in a range between 5.3 and 5.9, and the interest readers had all three of the highest scores that ranged from 6.3 to 7.0 (the scale maximum). Based on information gathered during our pre-interviews, we now turn to describing the children at each level of interest in reading. 

\subsubsection{Phase 4: Well-developed Individual Interest}
Of the three well-developed individual interest readers, one, MC8, had an extreme interest in reading. MC8 reads an exceptional amount (300 or more pages per day), and was described by his parent, P8, as ``off the charts''. He is such an avid reader that the local bookstore supplies him with ``a lot of advanced readers of books each month'' to have him ``give them an opinion on the books''. He described having very little need for a reading technology and struggled to find ideas of what one might do, because he feels no need to support improving his reading ability or motivation to read.

Our two other high interest readers, MC6 and MC7, described themselves as having high reading ability, and that they were an ``avid reader'', and ``always loved to read'', respectively. While they felt they were very good readers, they both indicated that they would like to continue to develop their reading ability. To support this development, they both thought a technology might suggest new books for them to read. MC6 also added a desire for a technology that would track their reading and provide some level of motivational support, while MC7 mentioned an interest in reading out loud.

\subsubsection{Phase 3: Emerging Individual Interest}
Our three emerging individual interest readers read between 30 to 60 minutes on an average day, and find that they can become engrossed in the right book. However, they often struggle to find new books that interest them, and are continually seeking out new material. For example MC4 said, ``I go out and try to find new books but I don't, like, find that many books appealing to me.'' This may be due to being reluctant to invest in reading new books, which leads them to re-read many of the books they already have several times. They value reading as a way to learn, they believe they are ``okay'' readers and would like to ``practice and get better'', and they appreciate ways of being supported in developing their ability. They also recognize a need for some external motivational support, particularly when they do not connect with a book right away (even if it is just a ``nagging mom''). MC3 did not suggest anything he would want a reading technology to do. However, both MC4 and MC5 would like a technology to help them understand and pronounce words while they read, and to give them feedback about the book as they read.

\subsubsection{Phase 2: Maintained situational interest}
Both of our maintained situational interest readers describe themselves as capable readers. While both will say they only ``like reading a little'', they do enjoy reading when they really like a book. However, they report different amounts of time spent reading. MC1 will read 30 minutes a day, because this is the goal her parents sets for her. MC2 reads very little outside of what is required for school. They both seem to feel that reading is something they should do more of, but don't prioritize this among other activities they prefer. Thus, they do not seek out reading unless required. For MC2 this conflict of feeling he should read more, but not really making the time for it appeared in our interview when describing that it is hard for him to find a book. He said, ``Yeah I am like looking right now. Well I am trying to get time to look for books. Well I mean I have got a lot of time but...'' Here he seems to be indicating that while he feels he should be looking for an interesting book, he is not actively doing so. Both children mention that tracking their reading with a log has been an effective method of motivation, although MC2 no longer uses any tracking, and both said this would be a welcome feature in a reading technology. MC1 also added that she would like a technology that would ``Listen to [her], and tell [her] what it thought of the book or what [they] read.'' MC2 indicated that he would like a technology that could support understanding by adding illustrations to depict what was happening in the text. Both of their parents mentioned motivational support as something they would like a technology to offer.

Having established each child's relationship with reading (see Table~\ref{tab:table1} for a summary), we will now begin describing themes that emerged from observations of the children interacting with the robot and their post-interviews.


\begin{table}[]
	\centering
	\begin{tabular}{cccc}
		& {\small \textbf{Phase}} & & {\small \textbf{FID-RS}} \\
		{\small \textbf{Participant}} & {\small \textbf{Number}}& {\small \textbf{Phase of Interest}}
		& {\small \textbf{Score}} \\
		\midrule		
		MC1  & 2       & maintained situational    & 4.5    \\
		MC2  & 2       & maintained situational    & 5.4    \\
		MC3  & 3       & emerging individual      & 5.4    \\
		MC4  & 3       & emerging individual       & 5.3    \\
		MC5  & 3       & emerging individual       & 5.9    \\
		MC6  & 4       & well-developed individual & 6.7    \\
		MC7  & 4       & well-developed individual & 7.0    \\
		MC8  & 4       & well-developed individual & 6.3   \\
		\midrule     
	\end{tabular}
	\caption{Phase of interest for main child participants. The FID-RS score is a measure of individual interest in reading. Note that no participants were found to be in phase 1.}
	\label{tab:table1}
\end{table}

\subsection{Development of Interest}

Our initial design of the learning companion robot was heavily influenced by interest development theory, and our analysis found that the children in our study describe much of their interaction in terms of the characteristics of interest development: \textit{positive affect} (i.e., enjoyment), \textit{value} for the interaction, and increasing ability or \textit{knowledge} as a result of the interaction. Positive affect was generally mentioned across all participants. All said they liked something about their interaction with Minnie, and described the interaction as ``nice'', ``fun'', and ``cool''. Positive affect was attributed equally across all the interaction features that the children mentioned.
	\begin{quote}
		\textbf{MC5:} I liked how she could recommend books and she commented during the book about different parts.
	\end{quote}
		
Phase 4 children were much less likely than those in phase 2 or 3, to talk about their interaction with the learning companion robot as a helpful or valuable experience. For those in phase 2 and 3, Positive affect and value were often attributed to specific aspects of the interaction with the learning companion robot, and these were fairly evenly distributed across all the features mentioned. There was one exception to this: tracking reading time and page number.
	
With only half of our participant families discussing it, the least frequently mentioned aspect of the interaction was Minnie's ability to track reading times and page numbers. Both MC2 and MC6 suggest that they ``like'' this feature and it was ``nice'', and three children across all phases of interest (MC2, MC4, MC6) refer to the robots ability to track their progress as ``helpful''. For example: 
	
	\begin{quote}
	\textbf{MC2:} It was helpful because that way if you take a break, he still knows what page you are on.
	\end{quote}
	
It is interesting that only families who refer to this type of tracking as something they would want a technology to do in pre-interviews, are the ones who discuss this after interacting with Minnie. Many families also had explained that this had been a way of helping to motivate their children, but none said that this feature offered any motivational support. It may be that our method of tracking was not conducive to increased motivation to read, but was rather only a nice and helpful feature to manage the process of reading.

In contrast to children referencing many different features of the robot interaction when describing positive affect and value, only two features were mentioned in reference to improving reading ability. Comments made by Minnie during the reading was thought to be a way to improve reading comprehension. Children felt that Minnie's comments were a way of ``recapping'' or helping them see the story from ``a different perspective''. Reading out loud was a feature of the interaction that children described as potentially improving their reading skills (e.g., vocabulary and pronunciation).

	\begin{quote}
		 \vspace{2mm}
		 \textbf{R:} You kind of talked about the difference about reading out loud. And that, you asked early if you should read out loud. Umm.. Did you like that experience?
		 		 
		 \textbf{MC4:} It kind of helped with some words, but at the same time, it didn't really help with other words.
		 
		 \textbf{R:} Okay, what do you mean by "helped"? Helped with what?
		 
		 \textbf{MC4:} Like, helped me, like, pronounce them better.
		  
		 \textbf{R:} Okay. Do you think, so is that, kind of, building...like is that making you better at reading you mean? Or...
		 
		 \textbf{MC4:} Uhh...yeah. 
		 
	\end{quote}

Also, describing the interaction with Minnie as a way to develop reading ability was almost entirely done by phase 3 readers. Our phase 2 and phase 4 children each only mention improving reading ability once, and seem to focus more on liking the learning companion robot and how the interaction was helpful to the process of reading, but not necessarily their ability to read. This may indicate that the support provided by Minnie is a good match for children who feel they are good readers, but need a little help with both comprehension and reading skill. When we asked our families about improvements or changes they would make to the learning companion robot, those in phase 2 focused on changing the interaction to better support reading ability, while those in other phases mostly mention ways of improving ease of use. 

	\begin{quote}
	 \textbf{R:} Ohh, ok, so there would that be something like... would you want if you were mispronouncing a word or like said something wrong would you want a robot to tell you that? Or would you rather the robot just kind of like let it go?
	 
	 \textbf{P1:} ((shakes head yes))
	 
	 \textbf{OC:} [She should] tell me that. 
	 
	 \textbf{R:} ((to MC1)) Would you wanna know?
	 
	 \textbf{MC1:} Yeah.
	 
	 \textbf{R:} Yeah, ok. Would that be like, when would you wanna know that? Would it be at the end of the whole thing or would it be right away while your reading?
	 
	 \textbf{MC1:} Right away. Like if it was a chapter book, at the end the chapter [it] would tell you all the words that you got wrong and [you] had to say them.
	\end{quote}

For our assessment of how interest may be developed as a result of the interaction, we consider how well the learning companion robot might be able to develop interest for children at differing phases of interest in reading. We find that Minnie, as designed, may be best suited for those in an emerging individual phase of interest (phase 3). These children expressed experiencing all three characteristics of interest development -- positive affect, value, and ability development -- as a result of their interaction with Minnie, and support across all these areas is important for developing interest at this phase. Minnie may also be appropriate for our maintained situational interest (phase 2) and well-developed individual interest (phase 4) participants --   although, this most likely does not pertain to those at extreme levels of interest (i.e., MC8). Their interest in reading did seem to be triggered from their enjoyment of the interaction and the value they found in working with Minnie -- both important factors in developing interest in a domain. However, some modifications to improve the learning companion's capacity for supporting the improvement of reading ability at higher and lower levels are needed. Suggestions for future improvement from our families suggest this might include ``chapter summaries'' or ``comprehension questions'' for our phase 2 readers, and the option to read silently while still scanning for Minnie's comments or ``connecting outside of the book to other places'' for phase 4 readers. 

\subsection{Attribution of Human Characteristics}

In order to best promote reading engagement and interest, we deliberately designed our learning companion robot to act in a social way. One important indicator that this goal was achieved came from our participants' attributing several human characteristics to Minnie. The families described the robot as being able to ``look'', ``listen'', ``talk'' and ``understand'', as well as having personality characteristics such as being ``jolly'', ``pleasant'' and ``funny''. They also attributed thoughts and feelings to the robot such as feeling ``happy'', making ``connections'' to what you were reading, that Minnie ``liked the book'', and ``had something to say''. When asked why they got this impression, many referenced the feedback and comments, as well as the physical movements made by Minnie. 

Notions about Minnie's feelings and personality came from both physical and software design. Specifically, Minnie's physical shape (e.g., enlarged rounded head and large eyes) seemed to influence perception of her personality, and it's speech capabilities that were tied to the specific context seemed to give the appearance of thoughts and feelings. 

One feature of the learning companion robot design that had a significant impact on this impression of human characteristics appeared to be the continuous face tracking during the interaction. Not only was the robot programmed to orient its head towards each face detected, but to also include intervals of randomized movements in other directions. MC6 noted that it ``was staring at you'', and while MC1 worked with Minnie, her sibling, sitting near them, noted that Minnie would point her head directly at each of them.

	 \begin{quote}
	 	 \textbf{OC1:} It like looked around the room and saw stuff, and it was looking at me and MC1 a lot.
	 \end{quote}
	
Feeling that Minnie could understand was often attributed to what the learning companion robot would say. Families felt that Minnie ``knew your name'', because she would use their name when saying hello. They also saw her comments during the book to be an indication that she could comprehend the story.

	\begin{quote}
		\textbf{MC6:} She talked about the whole thing you've read and the chapter, cause after I read how someone in one book put up a sign on their door she asked like `Oh I wonder why was there a sign on the door?'
	\end{quote} 

Based on prior work in human-robot interaction, we did anticipate the attribution of human characteristic to the robot.  This is a somewhat common way of creating social robots, and it was intentionally used in order to further our goals in designing the robot as a learning companion.  However, this was thought of as a leverage mechanism towards these further ends.  We were surprised to find that the social interaction tapped into a desire for social companionship while reading.

\subsection{Social Companionship}

The combination of the social behavior of the robot with many of the children's desire to socially engage while reading, produced an effect that we describe as \textit{social companionship}. We did not expected the children in the study to express a desire to socially engage while reading.  However, this idea came up with most of our families, with the exception of MC3 and MC8, during our interviews.  For the families who spoke about reading \textit{with} someone, a lack of social companionship was cited as a reason that reading could be boring, and reading together was a habit that many of our families currently engaged in.  Since Minnie was programmed to behave socially, this behavior seemed to help children find social companionship while they read with it. Not only was social companionship was positively discussed by almost all of our families during our interviews, but it appeared to be beneficial across all phases of interest. 

Both of our phase 2 readers seemed to feel that this helped them engage with the reading activity.  MC1 had complained that reading is usually boring, but she said Minnie ``encouraged me'' and reading with it was enjoyable because there was someone else there.
	
	\begin{quote}
		\textbf{R:} Did you enjoy working with Minnie?
		
		\textbf{MC1:} ((shake of head yes))
		
		\textbf{R:} What was nice about it?
		
		\textbf{MC1:} ((pauses to think))I liked the company?
		
		\textbf{R:} Ok, tell me a little about that, what do you mean?
		
		\textbf{MC1:} ((pauses to think)) I don't know. I usually read alone so it was like different.
	\end{quote}

 Our other phase 2 reader, MC2, also seemed to find having Minnie there as a social companion while reading to stimulate his motivation to read.  
  
	 \begin{quote}
	   \textbf{MC2:} Because I mean she helps, it is pretty much like having someone there helping you read. So like if I am home alone, I can have it sit out and I can, it can help me with my stuff while I am reading. 
	   
	   \textbf{R:} So, I will come back and I wonder then, so you said a lot of time with reading, you don't really have time for it. Do you think Minnie would make you feel, would you make more time? 
	   
	   \textbf{MC2:} Yes. 
 	\end{quote}
 
 The social companionship of reading with Minnie also seemed to be a positive experience for our phase 3 and 4 readers as well.  Although, this took different forms.  MC4 liked that reading with Minnie felt like to reading to someone younger.
 
 \begin{quote}
 	 \textbf{MC4:} It was a fun change. 
 	 
 	 \textbf{R:} Okay. How so? What do you mean by that?
 	 
 	 \textbf{MC4:} Umm like I found it, like, that...my, the voice in my head that I use when I'm reading to myself sounds different than when I read out loud. . . It felt like I was reading to a little kid. And, like, helping them understand some stuff.
 	 \end{quote}
 	 
MC5 related the experience to the positive feelings they had while reading with their family.

	\begin{quote}
		
		\textbf{R:} What did you like about it, what was nice?
		 
		 \textbf{MC5:} Um, like the it felt like it was another person listening and like I could communicate with her.
		 
		 \textbf{R:} I see. Ok. And I know you said like with your family you 	guys sort of do that. Did you get that same sort of sense of like you and Minnie were sort of like reading together and sort of sharing the book?
		 
		 \textbf{MC5:} Yeah.
	\end{quote}

MC6 found that the social companionship led to better reading.

	\begin{quote} 	
		 \textbf{MC6:} It . . .  kind of just listened to you instead of just you reading by yourself.
		 
		 \textbf{R:} How did the listening to you, how did that make your reading different?
		 
		 \textbf{MC6} Um, maybe because (2 seconds) cause you actually tried to do really good. Instead of just like.
		 
		 \textbf{R:} So, you maybe did a better job of reading because Minnie was listening to you, you said?
		 
		 \textbf{MC6:} Mhmm [yes]. 
 \end{quote}
 
 Finally, MC7 found that through the comments while reading, Minnie took on a social companion role similar to MC7's friends who read the same books as her.
 
	 \begin{quote}
	 	  \textbf{MC7:} Umm well, like, at the end of the, like, end of the thing when you scanned, like, she would, like, kind of comprehend what you were reading it seemed like. And she would, like, make connections just like a friend would do. 
	 	  
	 \end{quote}
 
As we mentioned, the social aspects of our learning companion robot were not intended to allow Minnie to provide social companionship.  We do find, however, that this was a common experience for our participants, and was an effect across all phases of interest.  There is also evidence that the added social companionship of reading with our learning companion robot addresses each of the major areas of interest development: increases in positive affect, value, and knowledge and skills in the domain.  

\section{DISCUSSION}

\subsection{Design Implications}

\subsection{Limitations}

\section{CONCLUSION}
Our use of the technology as a probe into the reading habits of young readers help us understand more about a learning companion...


\section{Acknowledgments}
Removed for blind review.

\balance{}

\bibliographystyle{SIGCHI-Reference-Format}
\bibliography{chi17-michaelis}

\end{document}
